\documentclass{oci}
\usepackage[utf8]{inputenc}
\usepackage{lipsum}
\usepackage{tabularx}
\usepackage{booktabs}
\usepackage{cancel}

\title{Aplausómetro}

\begin{document}
\begin{problemDescription}
El equipo de producción del popular programa de televisión
Domingo Colosal quiere mejorar el sistema de selección
popular usado para determinar el ganador de sus concursos.
%
Su sistema actual consiste en que los participantes
son dispuestos en línea y el animador los presenta de
izquierda a derecha pidiendo al público que aplauda por ellos.
%
Después de presentar a todos los participantes, quién obtiene
la mayor cantidad de aplausos es considerado el ganador.

El problema con este sistema es que
un participante está en desventaja con respecto
al siguiente, pues luego de que el público aplaude por
alguien, la parte del público que quiere que gane el siguiente
sabe cuanto tiene que aplaudir para que gane.
%
Por lo tanto, si un participante recibe más aplausos
que el anterior, eso no quiere decir que sea más preferido.
%
Sin embargo, si un participante recibe menos aplausos
que el anterior, entonces si sabemos que este claramente
no debería ganar.

La producción cuenta con un nuevo aplausómetro digital
que mide la intensidad exacta de los aplausos en aplos
(la unidad internacional de intensidad de aplausos).
%
Aprovechando esta tecnología, la producción ha diseñado
un nuevo proceso de selección que se ejecuta en varias rondas.
%
En este nuevo proceso, el animador también presenta a cada uno
de los participantes de izquierda a derecha y pide al público
que aplaudan por ellos.
%
Si el número de aplos para un participante $i$ es estrictamente
menor que los aplos para el participante directamente a
su izquierda el participante $i$ queda eliminado.
%
Al final de la ronda, los participantes eliminados son sacados
de la fila y comienza una nueva ronda.
%
Notar que los participantes son sacados de la fila al finalizar
la ronda.
%
Notar además que el participante 1 nunca puede ser eliminado
de esta forma pues no tiene nadie a su izquierda.

El proceso termina cuando al finalizar una ronda
no hay ningún participante eliminado.
%
Notar que esto implica que el proceso termina cuando una
ronda parte con un solo participante, pues este no puede
ser eliminado si no hay más participantes.
%
Al finalizar el proceso, se declara como ganador a quién
tenga más aplausos de los participantes restantes o empate
en caso de haber más de uno con la misma cantidad.

Queda poco tiempo para el inicio de la nueva temporada
y la producción está preocupada de que ejecutar el nuevo proceso
tardará demasiadas rondas.
%
Dada la intensidad con que el público aplaude por cada participante,
y asumiendo que la intensidad del aplauso es siempre la misma por
cada participante, tu tarea es determinar la cantidad de rondas
necesarias para que el proceso termine.
\end{problemDescription}

\begin{inputDescription}
La primera línea de la entrada contiene un entero $N$
($1 \leq N \leq 10^6$) correspondiente a la cantidad
inicial de participantes.
%
Cada participante es numerado con un entero entre $1$ y $N$.
%
La segunda línea contiene $N$ enteros $a_1 a_2 \cdots a_N$.
%
El entero $a_i$ corresponde a la intensidad en aplos en que el público
aplaude para el participante $i$.
%
Sin importar la ronda, el público siempre aplaude $a_i$ aplos por el participante
$i$.
\end{inputDescription}

\begin{outputDescription}
La salida debe contender un entero correspondiente a la cantidad de rondas
necesarias para terminar el proceso.
\end{outputDescription}

\begin{scoreDescription}
  \subtask{40} Se probarán varios casos de prueba donde $N \leq 10^3$.
  \subtask{60} Se probarán varios casos de prueba sin restricciones adicionales.
\end{scoreDescription}

\begin{sampleDescription}
\sampleIO{sample-1}

% \hspace{2em}
% \begin{minipage}{0.95\textwidth}
\textbf{Explicación ejemplo:}
%
Como se muestra a continuación, en este ejemplo se
requieren cuatro rondas para terminar el proceso.
%
En la primera ronda se eliminan los participantes 3 y 4.
%
En la segunda se elimina al participante 5.
%
En la tercera se elimina al participante 6.
%
Finalmente, en la cuarta ronda no hay ningún eliminado y por
lo tanto el proceso termina al finalizar esta ronda.

\begin{center}
\begin{tabular}{llr cccccc}
\textbf{Primera ronda}& & Participante & 1 & 2 & \cancel{3} & \cancel{4} & 5 & 6 \\
                       & & Aplos        & 4 & 5 & \cancel{2} & \cancel{1} & 3 & 4 \\
\midrule
\textbf{Segunda ronda}& & Participante & 1 & 2 & \cancel{5} & 6 \\
                       & & Aplos        & 4 & 5 & \cancel{3} & 4 \\
\midrule
\textbf{Tercera ronda}& & Participante & 1 & 2 & \cancel{6} \\
                       & & Aplos        & 4 & 5 & \cancel{4} \\
\midrule
\textbf{Cuarta ronda} & & Participante & 1 & 2 \\
                       & & Aplos        & 4 & 5 \\
\end{tabular}
\end{center}

% \fcolorbox{gray}{white}{
%   \begin{minipage}[t][5.2cm]{0.45\textwidth}
%   \textbf{Primera ronda:}
%   \begin{itemize}
%   \item Participante 1: 4 aplos
%   \item Participante 2: 5 aplos
%   \item Participante 3: 2 aplos (eliminado)
%   \item Participante 4: 1 aplos (eliminado)
%   \item Participante 5: 3 aplos
%   \item Participante 6: 4 aplos
%   \end{itemize}
%   \end{minipage}
% }
% \fcolorbox{gray}{white}{
%   \begin{minipage}[t][5.2cm]{0.45\textwidth}
%   \textbf{Segunda ronda:}
%   \begin{itemize}
%   \item Participante 1: 4 aplos
%   \item Participante 2: 5 aplos
%   \item Participante 5: 3 aplos (eliminado)
%   \item Participante 6: 4 aplos
%   \end{itemize}
%   \end{minipage}
% }

% \fcolorbox{gray}{white}{
%   \begin{minipage}[t][3cm]{0.45\textwidth}
%   \textbf{Tercera ronda:}
%   \begin{itemize}
%   \item Participante 1: 4 aplos
%   \item Participante 2: 5 aplos
%   \item Participante 6: 4 aplos (eliminado)
%   \end{itemize}
%   \end{minipage}
% }
% \fcolorbox{gray}{white}{
%   \begin{minipage}[t][3cm]{0.45\textwidth}
%   \textbf{Cuarta ronda:}
%   \begin{itemize}
%   \item Participante 1: 4 aplos
%   \item Participante 2: 5 aplos
%   \end{itemize}
%   \end{minipage}
% }
% \end{minipage}


% \begin{tabularx}{\linewidth}{X X}
%   Primera ronda:
%   \begin{itemize}
%     \item Participante 1: 4 aplos
%     \item Participante 2: 5 aplos
%     \item Participante 3: 2 aplos (eliminado)
%     \item Participante 4: 1 aplos (eliminado)
%     \item Participante 5: 3 aplos
%     \item Participante 6: 4 aplos
%   \end{itemize}
%   &
%   Segunda ronda:
%   \begin{itemize}
%     \item Participante 1: 4 aplos
%     \item Participante 2: 5 aplos
%     \item Participante 5: 3 aplos (eliminado)
%     \item Participante 6: 4 aplos
%   \end{itemize}
% \\
% \hline
%   Tercera ronda:
%   \begin{itemize}
%     \item Participante 1: 4 aplos
%     \item Participante 2: 5 aplos
%     \item Participante 6: 4 aplos (eliminado)
%   \end{itemize}
%   &
%   Cuarta ronda:
%   \begin{itemize}
%     \item Participante 1: 4 aplos
%     \item Participante 2: 5 aplos
%   \end{itemize}
% \\
% \end{tabularx}



\sampleIO{sample-2}
\end{sampleDescription}

\end{document}
