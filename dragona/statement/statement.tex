\documentclass{oci}
\usepackage[utf8]{inputenc}
\usepackage{lipsum}

\title{Guerrera Dragona}

\begin{document}
\begin{problemDescription}
  Como todos sabrán, Anita fue recientemente
  declarada la nueva guerrera dragona, encargada
  de proteger del mal al reino de Ocilandia.
  Hace unos momentos, se iluminó una Anita-señal
  frente a la torre de $h$ pisos.

  ¡Resulta que el malvado Marcos nuevamente amenaza
  con destruir la OCI!

  Anita fue al rescate, pero se dio cuenta de que
  antes de poder derrotar a Marcos, debe subir
  hasta el último piso de la torre, enfrentándose
  contra su mayor enemigo en el proceso: las escaleras.

  Afortunadamente, la torre contiene $n$ sets de ascensores,
  pero como están en malas condiciones, el ascensor $i$
  solo sirve para subir desde el piso $a_i$
  $b_i$, sin parar entre medio. Cabe notar que estos ascensores
  no son también descensores (como la mayoría de ascensores),
  por lo que no es posible utilizarlos para bajar.

  Anita desea minimizar la cantidad de escaleras que
  necesita subir para alivianar su sufrimiento, sin embargo,
  evitará a toda costa bajar escaleras porque tiene
  problemas en las rodillas.

  Para poder planificar bien su viaje, necesita saber la
  cantidad mínima de escaleras que necesita subir.
  Normalmente lo calcularía mentalmente, pero está demasiado
  ocupada planificando cómo derrotar a Marcos, por lo que
  te pidió ayuda con esta tarea. ¿Podrás ayudar a Anita
  a llegar a Marcos?
\end{problemDescription}

\begin{inputDescription}
  La primera línea de la entrada contiene dos enteros
  $n$ ($0 \leq n \leq 10^5$) y $h$ ($1 \leq h \leq 10^5$)
  correspondientes a la cantidad de ascensores y la cantidad
  de pisos en la torre respectivamente.

  Luego siguen $n$ líneas describiendo cada uno de los ascensores.

  La i-ésima línea contiene dos enteros $a_i, b_i$
  ($1 \leq a_i < b_i \leq h$), que corresponde respectivamente
  al piso de inicio y al de llegada del ascensor.
\end{inputDescription}

\begin{outputDescription}
  La salida debe tener un único entero, correspondiente a la cantidad
  mínima de escaleras que Anita debe subir.
\end{outputDescription}

\begin{scoreDescription}
  \subtask{??} Se probarán varios casos de pureba donde $n \leq 1$.
  \subtask{??} Se probarán varios casos de prueba donde $b_i < a_{i+1}$ para todo $i$. Es decir, los
  ascensores están ordenados y no se intersectan.
  \subtask{??} Se probarán varios casos de prueba donde $n \leq 10^3$ y $h \leq 10^3$.
  \subtask{??} Se probarán varios casos de prueba sin restricciones adicionales
\end{scoreDescription}

\begin{sampleDescription}
\sampleIO{sample-1}
\sampleIO{sample-2}
\end{sampleDescription}

\end{document}
