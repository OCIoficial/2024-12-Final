\documentclass{oci}
\usepackage[utf8]{inputenc}
\usepackage{lipsum}

\title{Suma de ejemplo}

\begin{document}
\begin{problemDescription}
  Como todos sabrán, Anita fue recientemente
  declarada la nueva guerrera dragona, encargada
  de proteger del mal al reino de Ocilandia.
  %
  Hace unos momentos, se iluminó una Anita-señal
  frente a la torre de $h$ pisos.
  %
  Resulta que el malvado Marcos nuevamente amenaza
  con destruir la OCI.
  %
  Anita fue al rescate, pero se dio cuenta de que
  antes de poder derrotar a Marcos, debe subir
  hasta el último piso de la torre, enfrentándose
  contra su mayor enemigo en el proceso: las escaleras.

  Afortunadamente, la torre contiene $n$ sets de ascensores,
  pero como están en malas condiciones, el ascensor $i$
  solo sirve para subir desde el piso $a_i$
  $b_i$, sin parar entre medio.

  Anita desea minimizar la cantidad de escaleras que
  necesita subir para alivianar su sufrimiento, sin embargo,
  evitará a toda costa bajar escaleras porque tiene
  problemas en las rodillas.
  %
  Ella necesita que calcules la cantidad mínima de escaleras
  que necesita subir dada esta restricción.
\end{problemDescription}

\begin{inputDescription}
  La primera línea de la entrada contiene dos enteros
  $n$ ($0 \leq n \leq 10^5$) y $h$ ($0< h 10^5$)
  correspondientes a la cantidad de ascensores y la cantidad
  de pisos en la torre.
  %
  Luego siguen $n$ líneas describiendo cada uno de los ascensores.
\end{inputDescription}

\begin{outputDescription}
  La salida debe contener un único entero correspondiente a la suma de $a$ y $b$.
\end{outputDescription}

\begin{scoreDescription}
  \subtask{50} Se probarán varios casos de prueba donde $-10^9\leq a, b \leq 10^9$.
  \subtask{50} Se probarán varios casos de prueba sin restricciones adicionales.
\end{scoreDescription}

\begin{sampleDescription}
\sampleIO{sample-1}
\sampleIO{sample-2}
\end{sampleDescription}

\end{document}
