\documentclass{oci}
\usepackage[utf8]{inputenc}
\usepackage{lipsum}

\title{Guerrera Dragona}

\begin{document}
\begin{problemDescription}
  Como todos sabrán, Anita fue recientemente
  declarada la nueva guerrera dragona, encargada
  de proteger del mal al Valle de la Paz.
  %
  En su primera misión, nuestra heroína debe ir
  a la Torre de la Llama Sagrada, donde
  el malvado Marcos nuevamente amenaza con
  destruir todo a su paso.
  %
  Para su pesar, al llegar a la torre se dio cuenta
  de que debe primero subir hasta el último piso,
  enfrentándose a su viejo enemigo en el proceso: las escaleras.

  % ¡Resulta que el malvado Marcos nuevamente amenaza
  % con destruir la OCI!

  La Torre de la Llama Sagrada tiene $h$ pisos numerados
  de 1 a $h$.
  %
  Para cada piso $i$ ($1 \leq i < h$) existe una escalera que
  puede ser usada para subir al piso $i+1$.
  %
  Adicionalmente, la torre posee $n$ ascensores.
  %
  Estos están en muy malas condiciones y por lo tanto el
  ascensor $j$ solo sirve para subir desde el piso $a_j$ hasta el
  $b_j$, sin parar entre medio.
  %
  Además, cabe notar que estos solo son ascensores y no descensores,
  por lo que no es posible utilizarlos para bajar.

  Anita quiere minimizar la cantidad de escaleras que
  necesita subir para alivianar su sufrimiento, sin embargo,
  evitará a toda costa bajarlas porque tiene
  problemas en las rodillas.
  %
  Específicamente, usando los ascensores y las escaleras para subir,
  Anita quiere encontrar una ruta desde el piso 1 al $h$ que
  minimice la cantidad de escaleras que debe subir.
  %
  Normalmente lo calcularía mentalmente, pero está demasiado
  ocupada planificando cómo derrotar a Marcos, por lo que
  te pidió ayuda con esta tarea.
  %
  ¿Podrás ayudar a Anita a subir la torre para
  que enfrente a Marcos y así pueda salvar al Valle de
  la Paz?
\end{problemDescription}

\begin{inputDescription}
  La primera línea de la entrada contiene dos enteros
  $n$ y $h$ ($0 \leq n \leq 10^5$, $1 \leq h \leq 10^6$)
  correspondientes a la cantidad de ascensores y la cantidad
  de pisos en la torre respectivamente.

  Luego, siguen $n$ líneas describiendo cada uno de los ascensores.

  La $j$-ésima línea contiene dos enteros $a_j, b_j$
  ($1 \leq a_j < b_j \leq h$), que corresponden respectivamente
  al piso de inicio y al de llegada del ascensor $j$.
\end{inputDescription}

\begin{outputDescription}
  La salida debe tener un único entero, correspondiente a la
  cantidad mínima de escaleras con la que es posible ir desde
  el piso 1 al $h$.
\end{outputDescription}

\newpage
\begin{scoreDescription}
  \subtask{5} Se probarán varios casos de prueba donde $n = 1$, es decir, hay exactamente un
  ascensor.
  \subtask{20} Se probarán varios casos de prueba donde $b_j < a_{j+1}$ para todo $j$.
  Es decir, los ascensores están ordenados y no se intersectan.
  \subtask{20} Se probarán varios casos de prueba donde $n \leq 20$.
  \subtask{55} Se probarán varios casos de prueba sin restricciones adicionales.
\end{scoreDescription}

\begin{sampleDescription}
\sampleIO{sample-1}
\sampleIO{sample-2}
\end{sampleDescription}

\end{document}
