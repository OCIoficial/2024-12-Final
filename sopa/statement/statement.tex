\documentclass{oci}
\usepackage[utf8]{inputenc}
\usepackage{lipsum}
\usepackage{amsmath}

\title{Sopa de letras}

\begin{document}
\begin{problemDescription}
  En el Instituto de Colección de Palabras en Cuadrículas (ICPC) son fanáticos de los juegos que
  involucran palabras tales como la famosa sopa de letras, en donde el jugador debe buscar palabras
  ocultas en una cuadrícula de $n \times n$ casillas con letras.

  Cada año la ICPC organiza una convención donde se presentan innovadoras variantes de juegos de palabras,
  y este año no fue la excepción. Esta vez se anunció que traerían una nueva versión de 
  la sopa de letras llamada ``sopa de letras cilíndrica''. La sopa de letras cilíndrica funciona igual
  que una sopa de letras normal, solo que además la cuadrícula es colocada alrededor de un cilindro.

  En particular, las reglas de la sopa de letras cilíndrica son las siguientes:
  \begin{itemize}
    \item El juego consiste en una cuadrícula de $n \times n$ donde cada casilla tiene una letra
      escrita.
    \item Al jugador se le presenta además una lista de $m$ palabras las cuales pueden estar o no
      escritas en la cuadrícula. Para cada una de ellas, el jugador debe responder si está o no
      en la cuadrícula.
    \item Una palabra puede estar escrita en cualquier lugar de forma vertical de arriba hacia abajo, 
      o de forma horizontal de izquierda a derecha.
    \item La cuadrícula está colocada alrededor de un cilindro. Es decir, después de la última letra
      de cada fila viene la primera letra de la misma fila, y antes de la primera letra de cada fila
      viene la última letra de la misma fila.
  \end{itemize}

  Como ya has demostrado tus habilidades de lógica y programación clasificando a la final nacional de la OCI,
  decides hacer un programa que resuelva una sopa de letras cilíndricas y así impresionar a los jueces en la
  convención de la ICPC.
\end{problemDescription}

\begin{inputDescription}
  La primera línea de la entrada contiene el entero $n$ ($2 \leq n \leq 50$). 

  Cada una de las siguientes $n$ líneas contienen $n$ caracteres, describiendo
  cada fila de la cuadrícula de arriba hacia abajo.

  La siguiente línea contiene un entero $m$ ($1 \leq m \leq 50$).

  Cada una de las siguientes $m$ líneas contienen un string $s$ que representa
    una palabra a buscar en la cuadrícula ($2 \leq \text{largo de } s \leq n$).

  Todos los caracteres de la cuadrícula y de las $m$ palabras corresponden a letras mayúsculas
  del alfabeto inglés.
\end{inputDescription}

\begin{outputDescription}
  Para cada una de las $m$ palabras, imprime una línea que diga \texttt{PRESENTE} si la palabra está
  en la sopa de letras cilíndrica o \texttt{AUSENTE} de lo contrario.
\end{outputDescription}

\begin{scoreDescription}
  \subtask{50} Se probarán varios casos de prueba donde solo hay palabras escritas de forma vertical de arriba
  hacia abajo.
  \subtask{50} Se probarán varios casos de prueba sin restricciones adicionales.
\end{scoreDescription}

\begin{sampleDescription}
\sampleIO{sample-1}
\sampleIO{sample-2}
\end{sampleDescription}

\end{document}
